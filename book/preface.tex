\chapter*{Preface}

There are usually two sides to technology: using it and working on it.
Loosely, I call this driver's ed vs. auto shop. In the former we learn
to operate a car: parking, signaling, turning, minding the rules of the
road. In the latter we would learn to repair the various systems on
a car. Notice I say ``would learn.'' I've never taken auto shop. My
repair skills are limited. This is typical. More people need to know
how to operate a car than need to know how to work on one.

For the expert, shop is more interesting. The temptation to fixate on it
is strong. This definitely holds for books and courses about compilers.
This book takes the driver's ed approach exclusively. It is completely
distinterested in how finite automata and context free grammars work
theoretically and how they are implemented. Our goal will be to use
tools to make languages.

One of the important inovations in software development since the start
of my career has nothing to do with code. It is the
agile process. One of the key concepts of agile development is the Minimum
Viable Product (MVP). The idea is to build something that will run and
deliver some value to the users. After it is in the users' hands, revisit
the product to add features. That will be my approach here.

We will start with the smallest language I could conceive. It will only add
single digit integers. When it does so, it will always only print the
result to standard out. Then we will commence adding features to this
language. We will have variables, integer and floating point types -- including
type checking, if/else and while, and functions. Even at the end
we won't have a fully useful language, but we will have learned to build one.

The main tool of this course is ANTLR4 which was written by Terence Parr.
He wrote an excellent book about it called
{\it The Definitive ANTLR4 Reference}.
You will probably want to consult that, or the various forms of online
documentation including especially antlr.org.

You can use ANTLR4 with various languages including C, C\verb+#+, JavaScript,
and Python. Here we will use Java for the simple fact that I know
it well enough. If you are going to follow along, choose a language
that is comfortable for you.

Let's get started.
