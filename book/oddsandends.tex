\chapter{Odds and Ends}

\section{Unary Negation}

It is not right to include negative numbers directly, because the unary negation
operator can be confused with the subtraction operator. This causes problems
with expressions like this:

{\footnotesize
\begin{verbatim}
    8-7
\end{verbatim}
}

If we leave the minus sign in our definition of NUMBER, the lexer will
grab -7 as a NUMBER in that expression. This will result in a missing
token, because there is then no operator between 8 and -7. There are
two ways it could tokenize this, at '-' then '7' or '-7'. It chooses
the longest way. Also, the antlr lexer does not backtrack to see if other
tokens would work for the parser. The parse just fails.

Had we written this with some space, our original approach would work:

{\footnotesize
\begin{verbatim}
    8 - 7
\end{verbatim}
}

As long as there is a space between the minus sign and the number,
the lexer cannot combine them into a NUMBER token. Some languages would
leave this as a quirk to explain in the documentation. But, quirks like
that surprise programmers who then say mean things about the language.

The correct solution is to introduce a new expr to the grammar.

{\footnotesize
\begin{verbatim}
    expr : '-' expr                     # unaryNegation
         | expr op=(MUL|DIV) expr       # MulDiv
         ... As before
\end{verbatim}
}

This operator has extraordinary precedence. It will win over subtraction
if there is a dispute. But, it also separates the literal minus
so that it can only take one character. Now a minus must stand alone in
all cases.

Consider the above example. The numeral 8 is taken as the lexeme of the
NUMBER token. This leaves the parser ready to match only the MulDiv
or AddSub options, since those are the only ones that begin with expr.
Thus, it wants a binary operator and finds the minus sign and the seven
is take as the second operator of subtraction.

A minus sign will only be taken for unary negation when the parser is
looking for a fresh expression, not when it is in the middle of a rule
that needs a binary operator.

Exercise

Add a UnaryNegationNode class and a visitUnaryNegation method in
EvalVisitor.java to create them.

\section{Error Reporting}

When a programmer makes an error, or the data fed into a program causes
a problem we are going to throw an exception. The recipient of that
exception will say bad things about us if we don't provide enough
information to track down the issue.

In particular, when we need to report a compilation problem our user
will want the line number where the error lies, perhaps even the character
position. Antlr collects the exact position of each token as the lexer
runs. We can ask for that and store the information.

Every antlr context object has a getStart method which returns a Token
object. Those objects respond to getLine which returns the line number
in the source file where the Token was seen. Tokens also have a
getCharPositionInLine method to say exactly where the matched lexeme
started.

Finding the line is easier. For example, you could start visitAssign this way:

{\footnotesize
\begin{verbatim}
    public Node visitAssign(ExprParser.AssignContext ctx) {
        int lineNumber = ctx.getStart().getLine();
\end{verbatim}
    }

When this assignment goes awry it will be because of the ID. Since you know its
name, your user may not care to know exactly where it was seen on the line.
But, if you wanted to know, you could dig to find out.

{\footnotesize
\begin{verbatim}
    Token idToken = ctx.ID().getToken();
    int beginningCharOnLine = idToken.getCharPositionInLine();
\end{verbatim}
}

To retrieve the lexeme, we have already been calling getText. We can use
the length that string to determine the ending position of the ID when we
report an error like not defined.

Exercise

1. Add lineNumber attribute to the Node abstract class. Then add a constructor
   which takes and stores the value. Complete Node by adding a get method for
   the line number. Modify all of its children to receive the line number and
   pass it to the super constructor. That will require EvalVisitor to retrieve
   the line number each time it makes a Node.
   Update all throws statements to include the line number in their messages
   (except the ReturnEncounteredException thrown in ReturnNode).

2. Exercise 1 for the token position.
