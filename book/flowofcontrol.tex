\chapter{Flow of Control}

There are a lot of structures in modern languages for flow of control.
There are loops. C has the three statement for. Java has that and a
variant which allows iteration over a collection. Many languages
have while to loop until their condition is false. Some have the
variant do/while which always runs the loop once, then begins
checking the condition. Practially every language has if/else.
A lot of languages have switch variant which is basically a syntactic
abbreviation of a set of ifs. There are even more choices.

We will stick our theme of keeping things simple. Our language will
have one loop: while, and if/else (but not else if).

\section{Grammar Update for If and While}

At core, both if and while have a condition. If runs its block once
if the condition holds and the else block once if now. While does
the same conditional check, but repeatedly runs the block until the
check is false. In way, they are variations on the theme.

We'll implement the conditional first.

The COMPARISON token is a list of all those alligators from grade school Math
with a couple of others thrown in.

{\footnotesize
\begin{verbatim}
    COMPARISON : '==' | '!=' | '<' | '>' | '<=' | '>=' ;
\end{verbatim}
}

With that token added to our common lexer rules, we can define the rule.

{\footnotesize
\begin{verbatim}
    conditional : expr compare=COMPARISON expr ;
\end{verbatim}
}

Our conditionals will compare two expressions. Notice that I have named
the comparison that actually matches as compare for use in our code.

Now we can add if and while statement productions.

{\footnotesize
\begin{verbatim}
    statement : 'print' expr NEWLINE                    # printExpr
              | ID '=' expr NEWLINE                     # assign
              | 'if' '(' conditional ')' block else?    # ifStatement
              | 'while' '(' conditional ')' block       # whileStatement
              | NEWLINE                                 # blank
              ;
\end{verbatim}
}

Note how similar the if and while structures are. The blocks are required. Else
is optional. We are not supporting else if. Feel free to add that.

{\footnotesize
\begin{verbatim}
    else : 'else' block ;
\end{verbatim}
}

By having a separate rule for else, we just have a tidier grammar. We could have
included this in-line with the if production:

{\footnotesize
\begin{verbatim}
    | 'if' '(' conditional ')' block ('else' block)?
\end{verbatim}
}

That's more cumbersome and becomes prohibitive if the else rule allows else if.

We will require braces (because you really should always use them, unless you
use Python).

{\footnotesize
\begin{verbatim}
    block : '{' statement* '}' NEWLINE? ;
\end{verbatim}
}

The statements are optional. If the user wants to do nothing, who are we to complain?
The newline is also optional after a brace so that people can write with cuddled braces.

{\footnotesize
\begin{verbatim}
    if (x < 5) {
        y = 2
    } else {
        y = 4
    }
\end{verbatim}
}

I don't prefer that style, but I can support with a question mark. I am to please.
But, we can't actually let the user have an if on a single line, since the other
statements use newlines as terminators. Language designers spend a lot of time
pondering details like these.

\section{Nodes for Conditionals, If, and While}

We need three new nodes for our internal tree. The others will need to help out
by expanding to accomodate a new method. The new nodes are IfNode, WhileNode,
and ConditionalNode.

The new method in Node.java is

{\footnotesize
\begin{verbatim}
    public abstract boolean getBooleanValue();
\end{verbatim}
}

Only the ConditionalNode will implement this in a meaningful way. The others
will have to come along for the ride. This is a theme of internal trees: uniformity
of structure with plenty of unused elements.

{\footnotesize
\begin{verbatim}
    public class ConditionalNode extends Node {
        Node left;
        String comparitor;
        Node right;
    
        public ConditionalNode(Node left, String comparitor, Node right) {
            this.left = left;
            this.comparitor = comparitor;
            this.right = right;
        }
    
        @Override
        public void act() {
        }
    
        @Override
        public Integer getIntValue() {
            return getBooleanValue() ? 1 : 0;
        }
    
        @Override
        public boolean getBooleanValue() {
            Integer leftValue = left.getIntValue();
            Integer rightValue = right.getIntValue();
    
            switch (comparitor) {
                case "==" :
                    return leftValue == rightValue;
                case "!=" :
                    return leftValue != rightValue;
                case "<" :
                    return leftValue < rightValue;
                case ">" :
                    return leftValue > rightValue;
                case "<=" :
                    return leftValue <= rightValue;
                case ">=" :
                    return leftValue >= rightValue;
            }
            return false; // unreachable, parser would have caught illegal comparitor
        }
    }
\end{verbatim}
}

When the visitor builds one of these nodes, it will pass in the left and right
expressions nodes along with the literal comparison operator. The getBooleanValue
method uses a switch on that to pick the right comparison operation.
All of the other nodes can return true or false as they wish. They cannot
legally appear in an if or while condition. In some languages they could (like Perl).
Then they would need to decide what truth means for them.

{\footnotesize
\begin{verbatim}
    public class IfNode extends Node {
        Node conditional;
        Node thenBlock;
        Node elseBlock;
    
        public IfNode(Node conditional, Node thenBlock, Node elseBlock) {
            this.conditional = conditional;
            this.thenBlock = thenBlock;
            this.elseBlock = elseBlock;
        }
    
        @Override
        public void act() {
            if (getBooleanValue()) {
                thenBlock.act();
            }
            else if (elseBlock != null) {
                elseBlock.act();
            }
        }
    
        @Override
        public Integer getIntValue() {
            return conditional.getIntValue();
        }
    
        @Override
        public boolean getBooleanValue() {
            return conditional.getBooleanValue();
        }
    }
\end{verbatim}
}

If statements have a condtional, a then block, and an optional else block.
To act, an if statement evaluates it conditional. If it is true, it runs the
then block. Otherwise, it runs the else block, but only if there is one.

You might be thinking that I missed a node. What about the block? That is
just a list of statements. We already have ProgramNode for that. I block
is just a miny program that only runs under the control of some other statement.

{\footnotesize
\begin{verbatim}
    public class WhileNode extends Node {
        Node conditional;
        Node block;
    
        public WhileNode(Node conditional, Node block) {
            this.conditional = conditional;
            this.block = block;
        }
    
        @Override
        public void act() {
            while (getBooleanValue()) {
                block.act();
            }
        }
    
        @Override
        public Integer getIntValue() {
            return conditional.getIntValue();
        }
    
        @Override
        public boolean getBooleanValue() {
            return conditional.getBooleanValue();
        }
    }
\end{verbatim}
}

A while node is strikinginly similar to an if node. The difference is that
it has only one block and it keeps running it until the condition is false.

\section{Adjustments to the Visitor}

I mentioned earlier that we can reuse the ProgramNode for blocks. We do need to
make some adjustments. Namely, we need to pull the code out of the visitProgram
method into a helper that visitBlock can share.
In EvalVisitor

{\footnotesize
\begin{verbatim}
    @Override
    public Node visitProgram(ExprParser.ProgramContext ctx) {
        return buildProgramNode(ctx.statement());
    }

    private Node buildProgramNode(List<ExprParser.StatementContext> parserStatements) {
        List<Node> programStatements = new ArrayList<>();
        for (ExprParser.StatementContext statement : parserStatements) {
            Node programStatement = visit(statement);
            if (programStatement != null) {
                programStatements.add(programStatement);
            }
        }
        return new ProgramNode(programStatements);
    }

    // ...
    @Override
    public Node visitBlock(ExprParser.BlockContext ctx) {
        return buildProgramNode(ctx.statement());
    }
\end{verbatim}
}

The other changes in EvalVisitor are visit methods to build the new nodes.

{\footnotesize
\begin{verbatim}
    @Override
    public Node visitIfStatement(ExprParser.IfStatementContext ctx) {
        Node conditional = visit(ctx.conditional());
        Node thenBlock = visit(ctx.block());
        Node elseBlock = null;

        if (ctx.else_() != null) {
            elseBlock = visit(ctx.else_());
        }

        return new IfNode(conditional, thenBlock, elseBlock);
    }
\end{verbatim}
}

Since else is marked optional in the grammar (with the question mark), antlr
names its lookup method with a trailing underscore: \verb+else_+.
We need to make
sure this is not null before visiting it. It is fine to give the IfNode constructor
a null for the else block, it checks for that before trying to run it.

Else has its own rule, so we need to implement its visit method.

{\footnotesize
\begin{verbatim}
    @Override
    public Node visitElse(ExprParser.ElseContext ctx) {
        return visit(ctx.block());
    }
\end{verbatim}
}

Remember that we don't visit it if it didn't collect anything. We only need
to check in visitIf.

Because while statements have only a required block, their visit method is simpler.

{\footnotesize
\begin{verbatim}
    @Override
    public Node visitWhileStatement(ExprParser.WhileStatementContext ctx) {
        Node conditional = visit(ctx.conditional());
        Node block = visit(ctx.block());

        return new WhileNode(conditional, block);
    }
\end{verbatim}
}

Merely visit the children and pass their nodes to the constructor.

Finally, we need visitConditional.

{\footnotesize
\begin{verbatim}
    @Override
    public Node visitConditional(ExprParser.ConditionalContext ctx) {
        String comparitor = ctx.compare.getText();
        Node left = visit(ctx.expr(0));
        Node right = visit(ctx.expr(1));

        return new ConditionalNode(left, comparitor, right);
    }
\end{verbatim}
}

That's a lot like a simpler version of visitAddSub. All we need to do is collect
the values from the children and pass them to the ConditionalNode constructor.

\section{Summary}

Now we can run programs like this one that calculates overtime pay.

{\footnotesize
\begin{verbatim}
    hrs = 42
    rate = 10
    pay = 0

    if (hrs < 40) {
        pay = hrs * rate
    }
    else {
        pay = 40 * rate + (hrs - 40) * rate * 3 / 2
    }

    print pay
\end{verbatim}
}

This figures pay with time and a half for overtime. Note that we
cannot multiply by 1.5 because we don's support floating point math.

Even though we can't print ``Lift Off!'' because we don't support strings,
we can count down.

{\footnotesize
\begin{verbatim}
    x = 10

    while (x > 0) {
        print x
        x = x - 1
    }
\end{verbatim}
}

\section{Testing Programs in Files}

We want to be able to test programs like these. There are lots of things
that we could test. We could build the Node tree by calling visit on
an EvalVisitor object. I'm going to check the final value of the interesting
variables.

Step one is to create

{\footnotesize
\begin{verbatim}
    src/test/resources/conditional.program
\end{verbatim}
}

with the contents of the overtime pay calculation above.

Then, run that program in a test. Finally, verify that the calculated
pay is 430.

{\footnotesize
\begin{verbatim}
    import org.antlr.v4.runtime.*;
    import org.antlr.v4.runtime.tree.*;
    
    import org.apache.commons.io.IOUtils;
    
    import org.junit.*;
    import static org.junit.Assert.*;
    
    public class EvalVisitorTest {
        @Test
        public void ifTest() throws Exception {
            String program = IOUtils.toString(
                this.getClass().getResourceAsStream("conditional.program"),
                "UTF-8"
            );
            CharStream        input  = CharStreams.fromString(program);
            ExprLexer         lexer  = new ExprLexer(input);
            CommonTokenStream tokens = new CommonTokenStream(lexer);
            ExprParser        parser = new ExprParser(tokens);
            ParseTree         tree   = parser.program();
    
            EvalVisitor eval = new EvalVisitor();
            Node nodeTree = eval.visit(tree);
            nodeTree.act();
    
            assertEquals(Integer.valueOf(430), eval.resolve("pay"));
        }
    }
\end{verbatim}
}

We need the same imports as the Runner. IOUtils makes it easy to
ingest the program file from the test resources directory. In the
test, getResourceAsStream will look in src/test/resources for the file
by name.

Everything after that is just like the Runner run method until we
reach the assertion. Then we require that 430 be the value in the
symbol table for the pay variable.

Exercises

1. Factor out the symbol into its own class. Adjust the Node classes that were
   receiving EvalVisitor in order to use the symbol table so they use the new
   class. Adjust the test of overtime pay so it can still inspect the final
   answer.

2. Add a do/while statement which always runs its block once, but then
   begins checking the conditional to decide whether to run again.
   Add a test to check at least one example of while or do/while.

3. Add else if to the grammar. Feel free to introduce a token for this like
   elif (used in Pyton) or elsif (used in Perl).
   Add a test with at least on else if structure.
