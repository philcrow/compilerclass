\chapter{Types}

So far our language has only worked with integers. This is
an obvious problem. It would be more useful if it worked
exclusively with floating point numbers. That would still
leave it limited. It is time to talk about types.

We actually introduced a parallel type for functions in
the last chapter. Anything with a name resolved at runtime
should be part of a single type system. Leaving functions separate
seemed to make our lives easier. Now we will incorporate
them into the type system along with ints and floats.

We couldn't expand the scheme of making a separate symbol
for floating point numbers, like we initially did for functions.
The problem is that the arithmetic operators
are shared by integer and floating point math. Since
those two must share, we should make a comprehensive
system for all symbols.

Some language introduce new operators to simplify the type
system. Perl has separate comparison operators for strings
and a separate symbol for concatentation, reserving the
plus sign for actual Math. But, addition of numbers still
works with both ints and floats.
If a single operator will work with two disperate kinds
of data, you must make it part of a type system.

\section{Floating Point Numbers}

As a first step we need to define what we mean by a floating
point number. How we will recognize it with a token?
Here are some of the things we want to be floating point numbers:

{\footnotesize
\begin{verbatim}
    1.2
    0.2
    45.
    .15
\end{verbatim}
}

For now, we'll avoid numbers that need exponents. Many languages
include things like 6.02e23. We'll leave that as an exercise.
We could allow negative versions of these as well. But see
the section on Unary Negation in the Odds and Ends chapter
for what we really need to do.

When you have lots of choices you should immeidately think
of the as alternative operator: |. We have used that
extensively in grammars. It works in regular expressions too.
Let me list the pieces along with what they match. Then,
we'll combine them.

\begin{tabular}{l l l}
To Match  & Use                    & In English \\
.15       & '.' [0-9]+             & a dot, then one or more digits \\
0.2       & '0' '.' [0-9]+         & a zero, a dot, and one or more digits \\
14.12     & [1-9][0-9]* '.' [0-9]* & any digit by zero, any number of digits, \\
45.       &                        & a dot, more optional digits \\
\end{tabular}

Putting these together we have a new lexer definition:

{\footnotesize
\begin{verbatim}
    FLOAT: '-'? ( '.' [0-9]+ | '0' '.' [0-9]+ | [1-9][0-9]* '.' [0-9]* )
\end{verbatim}
}

For now, we'll keep the possibiity of the numbers themselves being
negative. (Again, we really need unary negation.) Also, we have
no limits here. At some point during compilation, we need to notice
if a literal number is two big to fit in storage and raise an exception.
We also need to report an error if runtime computations result
in a number that is too big for storage.

Once we have a definition of what will count as a floating point
number, it is trivial to add them to the grammar. It is just another
expression alternative.

{\footnotesize
\begin{verbatim}
    expr : ...
         | FLOAT        # float
\end{verbatim}
}

This adds a visitFloat method to the parse tree walk. We'll need FloatNode
to go with that.

\section{A Simple Type System}

Naively allowing FLOATs as expressions will not work. We need to make
structural changes to the grammar and later to the parse tree walker,
symbol table, and nodes.

First, variables will need a type. This is foundational. When we first
encounter a variable, the programmer must tell us the type. Many languages
allow this type in a bare declaration. Java does this. It invites the
developer to use undefined variables, resulting in an error. Java calls
this a null pointer exception. You could prevent that by requiring
assignment during declaration. I will follow Java's garden path to
see how we need to report those errors later.

In addition to typing variables, we need types in two other places.
First, in the parameters to functions. Second, for the return values
of those functions. Here is the new definition of statement:

{\footnotesize
\begin{verbatim}
    statement : 'print' expr NEWLINE                            # printExpr
              | type ID NEWLINE                                 # declare
              | type ID '=' expr NEWLINE                        # declareAssign
              | ID '=' expr NEWLINE                             # assign
              | 'if' '(' conditional ')' block else?            # ifStatement
              | 'while' '(' conditional ')' block               # whileStatement
              | 'function' ID '(' paramList? ')' ':' type block # functionDefiniton
              | 'return' expr                                   # return
              | NEWLINE                                         # blank
              ;
\end{verbatim}
}

There are two new statement alternatives: one for bare declaration, one for
declaration with assignment. The function alternative has a new ':' type
addition for the return value. You could choose to use the Java approach
instead, by listing the type before the function name.

{\footnotesize
\begin{verbatim}
    type : 'int' | 'float' ;
\end{verbatim}
}

I chose the familar int and float types. They may confuse the Java developer,
because float will be implemented with Double. Since we are implementing in
Java that will simplify everything, since Double is the default float type
there. Unless we wanted to implement our own libraries to do the actual
calculations (or emit careful assembly) there is no use to fighting this.

Note that function does not appear in the type rule. We only allow
proper definitions with parameter lists.

The final alteration to the grammar is in the definition of paramList.
Now that parameters need a type, they need a rule.

{\footnotesize
\begin{verbatim}
    paramList : param ',' paramList
              | param
              ;

    param : type ID ;
\end{verbatim}
}

\section{Symbols}

With symbols that have types, we need a class to describe them. Like Nodes,
each type of symbol will inherit from this abstract class.

{\footnotesize
\begin{verbatim}
    public abstract class Symbol {
        String name;
        SymbolType type;
    
        public Symbol(String name, SymbolType type) {
            this.name = name;
            this.type = type;
        }
    
        public static Symbol getInstance(String typeName, String symbolName) {
            Symbol answer = null;
    
            switch (typeName) {
                case "int" :
                    return new IntSymbol(symbolName);
                case "float" :
                    return new FloatSymbol(symbolName);
                case "function" :
                    return new FunctionSymbol(symbolName);
            }
    
            return answer; // unreachable, the grammar will insist on a valid choice
        }
    
        public abstract Integer getIntValue();
        public abstract Double getFloatValue();
        public abstract FunctionNode getFunctionValue();
        public abstract void setIntValue(Integer newValue);
        public abstract void setFloatValue(Double newValue);
        public abstract void setFunctionValue(FunctionNode functionValue);
    
        public String getName() {
            return name;
        }
    
        public SymbolType getType() {
            return type;
        }
    
        public enum SymbolType {
            INT,
            FLOAT,
            FUNCTION
        }
    }
\end{verbatim}
}

At the bottom is the SymbolType. We make an entry there for
each type. The core facts of all symbols are then a name and a type.
This they share in the common parent class.
There are accessors for those attributes.

The subclass instances will know their values. They have accessors for
those values. If the type cannot meaningfully deliver one of the values,
it will throw an exception. If it has a choice, it should try to respond
with a good value. In particular, when an int is asked for a float
value, it should return its value as a float.

There is a factory here to create instances of the proper type. This
saves the main parse tree visitor from having to do that. There are
other reasonable designs.

The subclasses themselves are really just houses for a value.
Here is the class for ints. The others are similar.

{\footnotesize
\begin{verbatim}
    public class IntSymbol extends Symbol {
        Integer value;
    
        public IntSymbol(String name) {
            super(name, Symbol.SymbolType.INT);
        }
    
        public Integer getIntValue() {
            if (value == null) {
                throw new RuntimeException("Symbol '" + getName() +
                    "' used before definition");
            }
            return value;
        }
    
        public Double getFloatValue() {
            return Double.valueOf(value);
        }
    
        public FunctionNode getFunctionValue() {
            throw new RuntimeException("Symbol '" + getName() +
                "' is an int which cannot return a float");
        }
    
        public void setIntValue(Integer newValue) {
            value = newValue;
        }
    
        public void setFloatValue(Double newValue) {
            throw new RuntimeException("Symbol '" + getName() +
                "' is an int which cannot hold a float");
        }
    
        public void setFunctionValue(FunctionNode illegal) {
            throw new RuntimeException("Symbol '" + getName() +
                "' is an int which cannot hold a function");
        }
    
        public String toString() {
            return "int " + getName() + " " + getIntValue();
        }
    }
\end{verbatim}
}

\section{Symbol Tables}

The Symbol objects will live in a SymbolTable.

{\footnotesize
\begin{verbatim}
    import java.util.HashMap;
    import java.util.Map;
    
    public class SymbolTable {
        Map<String, Symbol> table;
        SymbolTable parent;
    
        public SymbolTable(SymbolTable parent) {
            table = new HashMap<String, Symbol>();
            this.parent = parent;
        }
    
        public static SymbolTable generateGlobalSymbolTable() {
            return new SymbolTable(null);
        }
    
        public SymbolTable getParent() {
            return parent;
        }
    
        public Symbol resolve(String name) {
            if (table.containsKey(name)) {
                return table.get(name);
            }
            else if (parent != null) {
                return parent.resolve(name);
            }
            else {
                throw new RuntimeException("No such symbol " + name);
            }
        }
    
        public void set(String name, Symbol symbol) {
            table.put(name, symbol);
        }
    
        public String toString() {
            StringBuilder sb = new StringBuilder();
            for (SymbolTable l = this; l != null; l = l.parent) {
                sb.append(l.table + "\n");
            }
            return sb.toString();
        }
    }
\end{verbatim}
}

The table itself is a Map, specifically a HashMap. When you construct a new
one, you pass in the parent. This allows us to chain the tables. More on
that in a moment.

For clarity, I created generateGlobalSymbolTable to make a parentless table.
That is for the interpreter to use for the global symbols. Other tables will
be temporary children during function execution.

When you are done with a function's table, the interpeter replaces the
current SymbolTable with its parent. To find that it calls getParent.

Calling resolve hands you the Symbol in the table. If it isn't there, the
parent is searched recursively. Only if the name is in none of the current
tables does the caller receive an exception. This is chaining, which allows
us to use global variables in functions. But, local variables in the block
of the function and its parameters will hide globals.

Java doesn't have globals in this sense. Yet, the same idea applies in instance
methods which use class attributes.

Set is simpler, since new variables always go in the top most table. (We
think of these tables as a stack.)

Notice that toString here walks back through the stack of tables so callers
can see everything. This is great for debugging.

\section{Interpreter Symbol Table Management}

In my approach, the interpreter knows all about the symbol tables. It keeps the
root table for globals and facilitates all actions on tables. These are the helpers.

{\footnotesize
\begin{verbatim}
public class EvalVisitor extends ExprBaseVisitor<Node> {
    SymbolTable symbols = SymbolTable.generateGlobalSymbolTable();

    public Symbol declare(String type, String id) {
        Symbol newSymbol = Symbol.getInstance(type, id);
        symbols.set(id, newSymbol);
        return newSymbol;
    }

    public Symbol resolve(String name) {
        return symbols.resolve(name);
    }

    public void setValue(String name, Integer newValue) {
        symbols.resolve(name).setIntValue(newValue);
    }

    public void setValue(String name, Double newValue) {
        symbols.resolve(name).setFloatValue(newValue);
    }

    public void setReturnValue(String name, Symbol symbol) {
        symbols.set(name, symbol);
    }

    public void pushCallStackFrame() {
        symbols = new SymbolTable(symbols);
    }

    public void popCallStackFrame() {
        symbols = symbols.getParent();
    }

    public void dumpSymbolTables() {
        System.err.println( "Symbols:\n" + symbols.toString());
    }
\end{verbatim}
}

The symbols variable always points to the top of the symbol table stack.
It starts with the parentless root for globals during class initialization.
Declaring a symbol uses the type name from the grammar: int, float, function
and the name. Even if the programmer is assigning during declaration, we
do that in two steps: declare then assign.

Resolve delivers the Symbol of the given name to the caller, or throws an
error if that symbol table is not in the current stack. Note that it is
up to the caller to notice if the the type of the symbol is illegal.
We could have a compilation pass between EvalVisit's walk and calling
act on the root program node.

The setValue methods allow the caller to provide the properly typed value
for the symbol. Mismatches ought to be checked somewhere.

The return statement has a special purpose helper so that it doesn't need
to declare the return variable. It always has the same name and it
could call declare and set like everyone else. But, this syntactic sugar
looked good to me.

To start a function call, CallNode will ask the interpreter to pushCallStackFrame.
This creates a new table, whose parent is the current one. It becomes the
current one. That parent link in the SymbolTable class is what makes it
capable of being a stack. This method affects the push.

When the CallNode is done with the function invocation, it must retrieve
the return value from the current symbol table, then discard that by
calling popCallStackFrame. That prevents symbols from bleeding
between function calls.

Finally, a nice way to call toString on the top symbol table is available
for anyone who has an interpreter instance handy: dumpSymbolTables.

\section{Visiting Methods and Nodes}

We need some visit method adjustments for the statements we introduced:
visitDeclare and visitDeclareAssign are brand new. We also need to adjust
visitFunctionDefinition due to the increased complexity in gathering the
parameters.

We also need to visitFloat, which is a lot like visitInt.

The CallNode needs changes to account for typed parameters.

\subsection{Visiting Methods}

Bare declarations are compile only. You don't need to store for those.

{\footnotesize
\begin{verbatim}
    @Override
    public Node visitDeclare(ExprParser.DeclareContext ctx) {
        String type = ctx.type().getText();
        String id = ctx.ID().getText();

        declare(type, id);
        return null;
    }
\end{verbatim}
}

We lookup the type and the id, then use the declare helper. There
is a bug here. Either we need to require values when variables
are declared in a function's block, or we need a way to note the
declaration so it can be added to the function's symbol table.

For declarations providing initial values, we do something similar,
but return an AssignNode.

{\footnotesize
\begin{verbatim}
    @Override
    public Node visitDeclareAssign(ExprParser.DeclareAssignContext ctx) {
        String type = ctx.type().getText();
        String id = ctx.ID().getText();
        Node valueNode = visit(ctx.expr());

        declare(type, id);
        return new AssignNode(id, valueNode, this);
    }
\end{verbatim}
}

Defining functions is also a compile time activity. We don't want them
in the runtime tree lest they run by accident. There several changes
to visitFunctionDefinition.

{\footnotesize
\begin{verbatim}
    @Override
    public Node visitFunctionDefiniton(ExprParser.FunctionDefinitonContext ctx) {
        List<Parameter> parameters = collectParameters(ctx.paramList());
        Node block = visit(ctx.block());
        String returnType = ctx.type().getText();

        FunctionNode functionNode = new FunctionNode(parameters, returnType, block);

        String functionName = ctx.ID().getText();
        declare("function", functionName);
        Symbol functionSymbol = resolve(functionName);
        // we need to put the node into the symbol table, not just record its name
        functionSymbol.setFunctionValue(functionNode);

        return null;
    }
\end{verbatim}
}

The collection of parameters is only cosmetically different here. The list is of
type Parameter instead of String. There is more work in collectParameters as we
will see below.

Collecting the block is the same. Now there is a return type too. These are
passed to the FunctionNode constructor. We again lookup the function's name,
but now we put it in the main symbol table with the variables. First, we must
declare it. Then we give a value by resolving the name to retrieve the
function's symbol and storing the node there.

\subsection{CallNode Changes}

Now that there are two types, callers could be expecting an int or a float
to come back from a function. As with other cases, we are happy to deliver
an int as a float, but using a float as an int is an error.

{\footnotesize
\begin{verbatim}
    import java.util.List;
    
    public class CallNode extends Node {
        String functionName;
        List<Node> arguments;
        EvalVisitor interpretter;
    
        public CallNode(String functionName, List<Node> arguments, EvalVisitor interpretter) {
            this.functionName = functionName;
            this.arguments = arguments;
            this.interpretter = interpretter;
        }
    
        @Override
        public void act() { }
    
        private Symbol callFunction() {
            FunctionNode function = getFunctionNode();
            interpretter.pushCallStackFrame();
    
            passParameters(function);
    
            try {
                function.invoke();
            }
            catch (ReturnEncounteredException e) {
                // expected, happens whenever the function hits a return statement
            }
    
            Symbol returnValue = interpretter.resolve(ReturnNode.RETURN_SYMBOL);
    
            interpretter.popCallStackFrame();
    
            return returnValue;
        }
    
        private FunctionNode getFunctionNode() {
            Symbol functionSymbol = interpretter.resolve(functionName);
            return functionSymbol.getFunctionValue();
        }
    
        private void passParameters(FunctionNode function) {
            List<Parameter> parameters = function.getParameters();
    
            // TODO verification of count match (correct number of args)
            for (int i = 0; i < parameters.size(); i++) {
                Parameter parameter = parameters.get(i);
                Node valueNode = arguments.get(i);
    
                // verify that the type on the parameter can be delivered by the valueNode
                if ("int".equals(parameter.getTypeName())) {
                    if (! valueNode.canBeInt()) {
                        throw new RuntimeException("parameter " + parameter.getName() + " to function " +
                                functionName + " must be int");
                    }
    
                    interpretter.declare("int", parameter.getName());
                    interpretter.setValue(parameter.getName(), valueNode.getIntValue());
                }
                else {
                    interpretter.declare("float", parameter.getName());
                    interpretter.setValue(parameter.getName(), valueNode.getFloatValue());
                }
            }
        }
    
        @Override
        public boolean canBeInt() {
            FunctionNode function = getFunctionNode();
            return function.canBeInt();
        }
    
        @Override
        public Integer getIntValue() {
            if (!canBeInt()) {
                throw new RuntimeException("Float function " + functionName + " asked to give an int");
            }
    
            Symbol returnValue = callFunction();
            return returnValue.getIntValue();
        }
    
        @Override
        public Double getFloatValue() {
            Symbol returnValue = callFunction();
            return returnValue.getFloatValue();
        }
    
        @Override
        public boolean getBooleanValue() {
            return false;
        }
    
        public String toString() {
            return "call " + functionName + "(" + arguments + ")";
        }
    }
\end{verbatim}
}

The attibutes, constructor and act methods are unchanged. That latter still does nothing.
Both getIntValue and getFloatValue work similarly. The only difference in structure
is that getIntValue complains if the function cannot return an int.
Both invoke callFunction to get the Symbol from the return statement in the function block.
Then they return the value from it.

It uses the getFunctionNode method to ask for symbol table lookup on the name of the function,
which it returns as the FunctionNode from that. Error checking is in order here.
Then, it asks the interpreter to add a symbol table to the stack. I called this
pushCallStackFrame because other languages will have other things to include
besides the symbol table. In particular if you are aiming for assembly, you
are going to need to include the return address so the program can keep running
from where it left off.

Once the new symbol table is in place, I need to passParameters into it.
I do that by walking two lists at once: the function's parameters and the
call's arguments. The types are compared and the usual complaint is raised
when a programmer asks to stuff a float value into an int slot.
The rest is just like variable declarations from assign statments.

With the parameters in place, it is time to invoke the function. We are expecting
the ReturnEncounteredException. The exception itself carries no information,
but it gives us back control at this point. Traditionally, statements would
be in sequential memory locations (for assembly). Then the return statement
would look in the top frame on the call stack for the line number to jump to.

The return statment puts the return value into the symbol table in a location
with an illegal name, to avoid colliding with programmer code. That symbol
name is \verb+ReturnNode.RETURN_SYMBOL+. Once we are holding that, it is time
to pop the symbol table off the stack and give back the value.

Exercises

1. Expand the alternatives in the regular expression for FLOAT to
   allow exponents. A lot of languages use the same notation.
   Visit https://www.json.org to see all the options. Keep in mind
   that in JSON number includes integers. Avoid those cases.

2. Fix the bug that puts variables declared without values in function blocks
   in the global symbol table. Either, tell the function node about
   the declaration so it can be put in the right table at run time,
   or call an error demanding that these declarations receive initial
   values.

3. Add a validation to ensure that a function has a return statement.

4. Add a void return type so functions can act as procedures. They
    would then only be useful for side effects, possibly printing.

5. Add better error messaging if the programmer attempts to call an
    undefined function.
