\chapter{Adding Digits}

Let's start with the simplest language I could think of. It will
add single digit integers. From this we will see quite a bit of
antlr. As we go, we'll add more features.

\section{A Program}

If we put this program into a text file:

{\footnotesize
\begin{verbatim}
    4+5
\end{verbatim}
}

we want our end result to be 9 printed on the screen.

\section{Structure}

Tidiness is essential for builing software. It is also more or less
enforced by tools like gradle. Gradle has a standard structure for
the directories where input files live. It is best not to fight the tool.

First make a directory for the work we are about to begin. I'll call mine
app. Then make this structure under that:

{\footnotesize
\begin{verbatim}
    app
    +-- src
        +-- main
        |   +-- antlr
        |   +-- java
        +-- test
            +-- java
\end{verbatim}
}

In the app directory, place a build.gradle file. You can use mine.
Look for me on github (https://github.com/philcrow). Look for compilerclass.
This book has sample code laid out as weeks in a class.
Warning: some of the weeks cover quite a lot of ground, so week might
be an optimistic term. Dig down to class.weeks/week1/app/build.gradle\footnote{
https://github.com/philcrow/compilerclass/blob/main/class.weeks/week1/app/build.gradle}

Notice that testing is going to be a key part of our effort. I'll feature
it all along.

Code that we feed to antlr will be under src/main/antlr. Code we write
ourselves will be under src/main/java. Tests will be in src/test/java.
I've chosen junit 4 for the tests, because I find it easier to configure
and use. There are other testing frameworks, including junit 5 (jupiter).
Choose one that suits you. But, do write tests. They save time in the end.

\section{Steps}

Now that we have places to put everything, we can start building a language.
In compilation, the first step is to recognize the little bits and pieces
that make up the language. These are called tokens. Once we know the
tokens, we can try to recognize them with a grammar. If that works we
will have a tree containing exactly what matched.

\section{Three Steps in ANTLR}

In antlr4, we can combine the tokens and the grammar rules in one file.
As things become more complex this becomes cumbersome (or impossible).
For now our we can create a single file called Calc.g4 under
src/main/antlr with this content:

{\footnotesize
\begin{verbatim}
    grammar Calc;
    program : NUMBER '+' NUMBER ;
    NUMBER : [0-9] ;
\end{verbatim}
}

As with Java, the name of the grammar on the first line (Calc) must
match the file name (Calc.g4).

There are two rules here. A NUMBER is a token (antlr knows this because
we capitalized its name. All caps is a convention, capitalization is required.).
You can define number in a lot of ways, all of the other ways will be
more interesting that this.

You specify the tokens with regular expressions. Use a search engine on
that term if you need advice.

Here, NUMBER is one of the numbers from 0 through 9. The brackets
make a class of characters so we can write them compactly. Inside the
brackets we can use a range, here 0-9. This will include all the
characters between 0 and 9 in the ASCII chart.

Each rule in anltr contains at least a name, a colon, a definition,
and a semi-colon.

Once we know what a number is, we can define a program in our language.
The rule for a program is: one number, a literal plus sign, and another
number. Again, the semi-colon is tell antlr that the rule is finished.

To make our language, we run a few commands at a command prompt:

{\footnotesize
\begin{verbatim}
    gradle build
    cd build/classes/java/main
    grun Calc program -gui -tokens -tree adder
\end{verbatim}
}

where adder is a text file with a legal program in our language like 4+5.
You may notice errors when this runs. We have some issues to resolve.
Thanfully, antlr is quite forgiving.

This will not print 9. We have not yet implemented the last step: running
the program. We have only parsed it.

In the picture that grun generates you can see the tree as antlr built
it from the grammar.

{\footnotesize
\begin{verbatim}
       program
         /|\
        / | \
       4  +  5
\end{verbatim}
}

This tells us that it saw a program. There are three elements to the
program, each becomes a child node of the root program node. These
are the tokens 4, +, and 5.

\section{Using ANTLR's Tree}

Above we asked grun to show us the tree that represents our program.
What antlr actually built was a set of files. Gradle put everything
antlr built in the bin directory. There are a lot of artifacts. If you
want rid of them you can type

{\footnotesize
\begin{verbatim}
    gradle clean
\end{verbatim}
}

That will remove the build directory.

Under build, we are most interested in the code antlr made for us to
use directly. This is found under

{\footnotesize
\begin{verbatim}
    build/generated-src/antlr/main
\end{verbatim}
}

There we see both helper files (which are of little interest) and
java files. The java files are

{\footnotesize
\begin{verbatim}
    CalcBaseListener.java
    CalcLexer.java
    CalcListener.java
    CalcParser.java
\end{verbatim}
}

Unless we are debugging problems with tokens, we won't need CalcLexer.
The other three need some explanation.

CalcListener.java is the interface for the tree listener. This will
have an enter and an exit method for each rule in the grammar whose
name starts with a lowercase letter (in other words everything but the
tokens).

When the grammar is complex, there are a lot of these methods. Implementing
this interface could require us to implement a lot of methods we don't
need. That is where CalcBaseListener.java enters. It implements the
interface with methods that don't do anything. By subclassing that,
we can override the methods we want and ignore the rest.

I called my subclass Interpretter.java. An interpretter is a program
the runs another program a line at a time. I initially created by copying
from CalcBaseListener.java. Then I renamed things and removed unwanted
things.

{\footnotesize
\begin{verbatim}
    import org.antlr.v4.runtime.ParserRuleContext;
    import org.antlr.v4.runtime.tree.ErrorNode;
    import org.antlr.v4.runtime.tree.TerminalNode;
    
    public class Interpretter extends CalcBaseListener {
        int answer = 0;
    
        @Override
        public void enterProgram(CalcParser.ProgramContext ctx) {
            int left = Integer.valueOf(ctx.NUMBER(0).getText());
            int right = Integer.valueOf(ctx.NUMBER(1).getText());
            answer = left + right;
        }
    
        public int getAnswer() {
            return answer;
        }
    }
\end{verbatim}
}

I kept the imports from CalcBaseListener.

Notice that exitProgram is not present. Don't implement methods that
don't need to do anything.

Each enter and exit method receives the ctx parameters which the
parser made for the tokens. It is the subtree for this node in the program
tree. That is, it represents the children of the rule in the enter method
name (program here).

It is useful to peek in the parser to see what methods you can call
on the context object. Look for the YourRuleContext class. Those
are the methods you can call. (Your IDE will know this and might
assist you.)

In our case, we know there is a plus sign (which is not available
since it is a required literal token). But, we are interested in
what to add together.

The methods are named from the grammar, so you can also use that
instead of digging in CalcParser itself. Here we want NUMBER.
Since there are two, we get a list.

Once we know which numbers were in the program, we need to add them.
You could stream here, but I've used an old school for-in loop instead.

Notice a couple of things. First, there is an instance member called
answer that is keeping track of the total. To see the answer, someone
will need to call getAnswer. Later we will learn about the visitor
pattern which makes this kind of answer collection both tidier and
a lot more flexible.

What remains is a driver to instantiate the Interpretter and invoke
the proper methods to obtain the answer. We actually want to do that
twice. Once for the operative code, so you can run programs. Once
in a test file, so gradle can perform our tests repeatedly as we go.

\section{Testing}

Some advocate writing tests first. They call this test driven development.
I find that a bit tedious, yet I know I want tests and write them as I
go. In order to easily test code, you need to structure it in a certain
way. This actually makes it easier to understand and maintain as a side
effect. Write tests as you go. Life will be better.

Even though I didn't write the test first, I'll show it first so you
can see the structure of what we are doing.

{\footnotesize
\begin{verbatim}
    import org.antlr.v4.runtime.*;
    
    import org.junit.*;
    import static org.junit.Assert.*;
    
    public class RunnerTest {
        @Test
        public void interpretterTests() {
            String program = "4+5";
            CharStream input = CharStreams.fromString(program);
            Runner runner = new Runner();
            int answer = runner.run(input);
            assertEquals(9, answer);
        }
    }
\end{verbatim}
}

There are includes for antlr, and junit. I'm using junit 4 since I have
had an easier time configuring and using it. Feel free to use any
other framework that suits your style.

In junit 4, we label methods that should be run by the test harness
with the @Test annotation. They should be public and void. After that,
they can do anything any other java code does.

This one starts by defining as complex a program as our current language
can understand. Once we have a string containing the program text, we
need to create a CharStream which antlr defines and expects. It is
brought in with the import of org.antlr.v4.runtime.*.

Runner is our class. It will have a run method that returns the answer
for the supplied program. The last line asserts that the answer should be 9.
We could add more tests. Some sad path (illegal syntax) tests would
be in order here.

\section{Driving}

The Runner will take of collecting the program supplied by the user,
passing it through the antlr processing methods, user our interpretter
and print the answer.

Again, the structure is helpful for testing. We don't want to just print
the answer. That would require the test to collect standard input which
is tedious at best.

{\footnotesize
\begin{verbatim}
    import org.antlr.v4.runtime.*;
    import org.antlr.v4.runtime.tree.*;
    
    public class Runner {
        public int run(String fileName) throws Exception {
            // 1. get the input into a stream from a file passed on command line
            CharStream input  = CharStreams.fromFileName(fileName);
            return run(input);
        }
    
        public int run(CharStream input) {
            // there are other ways to feed input, notably stdin
            // CharStreams.fromStream(System.in);
    
            // 2. user the stream to construct the lexer
            CalcLexer         lexer  = new CalcLexer(input);
    
            // 3. use the lexer to produce the token stream
            CommonTokenStream tokens = new CommonTokenStream(lexer);
    
            // 4. use the token stream to construct the parser
            CalcParser        parser = new CalcParser(tokens);
    
            // 5. tell the parser to start at the program rule
            ParseTree         tree   = parser.program();
    
            // 6. make a walker that can visit the nodes in the AST
            ParseTreeWalker   walker = new ParseTreeWalker();
    
            // Our custom code
            Interpretter  interpretter = new Interpretter();
            walker.walk(interpretter, tree);
    
            return interpretter.getAnswer();
        }
    
        public static void main(String[] args) throws Exception {
            String fileName = args[0];
            Runner runner = new Runner();
            int answer = runner.run(fileName);
    
            System.out.println( "answer: " + answer);
        }
    }
\end{verbatim}
}

In the main method, we expect the file name to be the only argument
supplied when a user runs the program. That file needs to contain
the program text. With the file name in hand, we can instantiate
a Runner object and call the run method that expects a file name.
When that returns, it will give us the answer to print.

The run method which accepts a file name uses CharStreams just like
the test did. Once it has a CharStream, it can call the other run
method remembering to return the answer. This is delegation.

The primary run method needs its input an a CharStream. It passes
that to the antlr Lexer. That is used to make a CommonTokenStream
which can be given to the parser. Once you have a parser, you can
call any rule in it by name. Here we only have one rule: program.
That makes the tree grun showed us earlier.

For the listener approach, we create a ParseTreeWalker and give it
our Intepretter (which must implement CalcListener) and the tree.
That will perform the traversal by calling the various enter and
exit methods.

At the end, we need to ask the interpretter for that answer. With
the listener approach, nothing is returned from the walk.

\section{summary}

We have now implemented one of the least useful languages in the
history of computing. Along the way we have seen a regular
expression to specify on digit integers, a grammar with only
two tokens (NUMBER and +), how to build that into a set of
really helpful code with antlr. Finally, we use the listener
pattern to interpret our language into an answer. We did not
forget to test.

There are obvious shortcomings with the language. It might be nice
to multiply, divide, or subtract. It might be nice to have numbers
with more than one digit. For a genuine language, we will want
flow of control, floating point numbers and other features.

There were also bugs while running this example. We never told
it what to do with whitespace.

Exercises

1. Add tests for input that won't work, see and verify errors.

2. Expand the NUMBER token definition in the grammar to allow
   multiple digits (try searching the web for regular expression).

3. Expand the grammar and interpretter to handle multiplication.
