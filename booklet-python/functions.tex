\chapter{Functions}

It is often said that functions\footnote{There are other similar
concepts: methods, subroutines, procedures. Our language will only
have the kind that can return one valus. Since we aren't implementing
objects, these are correctly called functions. But, the ideas apply
to the others.} are primarily to encourage re-use. In reality
we write them to name operations and move them out of the flow
of the program. This makes it easier to read and maintain the code
and gives us small points to test.

In any case, the function is a key feature of a good language. A way
to move control to a separate point and (optionally) get something
in return.

\section{Goal}

Now it is time to have some fun. I teased the syntax for our functions
in the symbol table chapter. Here is an example.

{\footnotesize
\begin{verbatim}
    fun fact(n) {
        if (n == 1) {
            return 1
        }
        else {
            return n * fact(n - 1)
        }
    }
    answer = fact(4)
    answer
\end{verbatim}
}

I have reiterated answer on a solo line so visitPrint will show the final
number. Assignments don't print.

\section{Grammar Additions}

There are two new statements in that short example: a function
definition and a return. Once the function is defined, we also
need a new expression to call it.

{\footnotesize
\begin{verbatim}
    statement: ...
             | 'fun' ID '(' params ')' block       # fun
             | 'return' expression NL              # return
\end{verbatim}
}

The fun keyword and the parentheses are literal. We will use
the same ID as for variables (lowercase letters, or whatever
improvement you have made on that). We will also use the block
defined for if/else and while.

For return, we have a new keyword and can reuse the expression rule.

That leaves only params in need of a definition.

{\footnotesize
\begin{verbatim}
    params: ID (',' ID)* ;
\end{verbatim}
}

This definition means that all of our functions will require at
least one parameter. You could make a different choice.\footnote{There
is an exercise to change this.}

Now we need a new expression to make a function call.

{\footnotesize
\begin{verbatim}
    expression: ...
              | ID '(' args ')'                       # call
\end{verbatim}
}

That requires a new rule for args. The ID will be the name from
the function definition.

{\footnotesize
\begin{verbatim}
    args: expression (',' expression)* ;
\end{verbatim}
}

This is analogous to the parameter list (and also requires at least
on thing to be passed to the function). But, these are values not names.
Even if we pass variable names, our call will resolve them to values
before giving them to the function. Some languages allow a pass by
reference semantic. Then the parameter becomes an alias for the variable
in question, allowing the function to alter its value as a side effect.
Our language is simpler than that.

\section{Implementation}

The first key feature of functions is that they don't run when we
first see them. Something else needs to call them. Thus, we need
to store each function definition. Plenty of languages treat functions
as symbols and put them in the standard symbol table. We could do that
here, but it wouldn't allow us to switch to an explicitly typed
language like Java. There a symbol table can only store one type and
we have already chosen integer for our variable.

\subsection{Storing Fun}

The simplest way to handle storing functions is to make a separate
dictionary. The key will be the name of the function. The value
will be a new Fun object described here.

In the Visitor constructor, add a dictionary

{\footnotesize
\begin{verbatim}
    class Visitor(CalcVisitor):
        def __init__(self):
            self.global_table = SymbolTable(None, None)
            self.symbol_table = self.global_table
            self.fun = {}
\end{verbatim}
}

In visitFunction we create a new Fun object and store
it.\footnote{I'm still returning an integer here with kind thoughts to
anyone implementing in language which demands that each visitor return
the same type, like Java.}

{\footnotesize
\begin{verbatim}
        def visitFun(self, ctx:CalcParser.FunContext):
            name = ctx.ID().getText()
            self.fun[name] = Fun(ctx.params(), ctx.block())
            return 0
\end{verbatim}
}

We'll see the Fun class a bit later. Once we have functions available,
we need two more things: return and call.

\subsection{Returning Fun}

When we encounter a return statement in a function, we need to compute
the returned value expression, store that in a convenient place,
and jump back to the calling location. In languages like assembly,
statements are numbered. When you make a call, you stash the number
of the calling location in the call stack frame. The return pulls that
out and uses it. We don't have that notion here.

We have choices for how to go back to the invocation point. The one
that seems simplest to me is to raise an exception. This is also in
keeping with a pythonic view of exceptions: they are just there
to help us control flow.

{\footnotesize
\begin{verbatim}
        def visitReturn(self, ctx:CalcParser.ReturnContext):
            answer = self.visit(ctx.expression())
            self.symbol_table.set("0retval", answer)
            raise  ReturnException()
\end{verbatim}
}

The value to return is obtained by visiting the expression, just as
we did in visitAssign. We need a place for the value to ride back to the
caller. I chose to hide it in the symbol table, by using an illegal
variable name. Our grammar will not allow the user to begin a variable
name with zero, so I can safely insert there.

When I raise my exception, I want it to be unique. I don't want the
caller to confuse a regular return statement with an actual problem
(like an undefined variable). So, I made a new exception class

{\footnotesize
\begin{verbatim}
    class ReturnException(Exception):
        pass;
\end{verbatim}
}

I inherit all behavior from the Exception base class and have nothing
to do in my class. Python requires me to explicitly say there is nothing
to do with its pass statement. The only step left is to import this
into the Visitor:

{\footnotesize
\begin{verbatim}
    from ReturnException import ReturnException
\end{verbatim}
}

\subsection{Calling Fun}

{\footnotesize
\begin{verbatim}
        def visitCall(self, ctx:CalcParser.CallContext):
            name = ctx.ID().getText()
            fun = self.fun[name]
            args = ctx.args()

            self.symbols = self.symbol_table.addFunctionScope(self.global_table)
            answer = fun.call(self, args)
            self.symbols = self.symbol_table.discardFunctionScope()
            return answer
\end{verbatim}
}

When a call expression is reached, it needs to retrieve the function's
name (using ID on the context object). Then retreive the fun from the
dictionary. It also needs to grab the context object for the args
from the CallContext.

Before the call, I need to have the symbol table make a new scope so
the function can't see the caller's variables unless they are global.
After the call, I need to restore the original symbol table.

The Fun class has a call method that does the work and delivers the
answer. I need to store and return that so the calling expression can
use it.

\subsection{The Fun Class}

We've now seen the API for the Fun class. It needs a constructor which
receives and stores a ParamsContext object and a BlockContext object.
It has a call method to invoke the function. That receives the Visitor object
and the ArgsContext object. The call method will populate the newly constructed
symbol table with values for the named parameters from
the arguments. Once the symbols have their values, it can visit the block.

{\footnotesize
\begin{verbatim}
    import sys
    from ReturnException import ReturnException

    class Fun:
        def __init__(self, params_ctx, block_ctx):
            self.param_ctx = params_ctx
            self.block_ctx = block_ctx
\end{verbatim}
}

When things go wrong, I want to exit the program with an error. I could
raise an exception, but that spews a stack trace. When I am ready to
return, I will raise a ReturnException, so I need to import it.

The constructor stores the received parameters.

The call method places the arguments into the symbol table using the
formal parameter names.

{\footnotesize
\begin{verbatim}
        def call(self, visitor, args_ctx):
            param_names = self.get_param_names()
            arg_values = self.get_arg_values(visitor, args_ctx)
            param_len = len(param_names)
            arg_len = len(arg_values)
            if param_len != arg_len:
                print(f"param count {param_len} != {arg_len}")
                sys.exit(1)

            for param, arg in zip(param_names, arg_values):
                visitor.symbol_table.set(param, arg)

            try:
                visitor.visitChildren(self.block_ctx)
            except ReturnException:
                pass
            return visitor.symbol_table.resolve("0retval")
\end{verbatim}
}

I'll show the \verb+get_param_names+ and \verb+get_arg_values+ later.
Call verifies
that there are the same number of parameters and arguments. If not,
it proclaims the error and exits. Again, that prevents python from
spewing a stack trace.

Python has a convenience builtin, zip, which allows us to walk
two lists at once. For each one, I use the symbol table set method
to install the parameter's value.

Once the parameters are in place, I just asked the visitor to
visit the children of the block context. That will execute all the
statements. I did this in a try block, because I'm expecting the
ReturnException.\footnote{It will actually be a fatal error not
to have a return statement when the last line in call tries to
resolve the 0return variable.} Any other exception is unhandled
and will halt the program.

Calling the function returns the value in 0retval, which a
return statement placed there.

Getting the list of parameter names from the ParamsContext and
getting the values of the arguments from the ArgsContext is similar work.

{\footnotesize
\begin{verbatim}
        def get_param_names(self):
            param_names = []
            for param in self.param_ctx.ID():
                param_names.append(param.getText())
            return param_names

        def get_arg_values(self, visitor, args_ctx):
            arg_values = []
            for arg in args_ctx.expression():
                arg_values.append(visitor.visit(arg))
            return arg_values
\end{verbatim}
}

Both methods make an empty list to return. They each walk a list
given by a context object with for. They each append to their list and
return it. The difference is in how what they put in the list.
A ParamsContext has an ID method. You can call that with an index to
get a specific ID from the list. But, if you call it without arguments,
the context returns a list of all the ID tokens. These have a getText
method that returns the name we want. For the args, we just ask
the visitor to visit each arg. That will evaluate whatever expression
is in that arg position.

\section{Summary}

We have now completed a language which supports arthimetic, variables,
conditional logic, looping, and functions. There are a lot more features
we could implement. These include: arrays, typed variables (at least
separate integers from floating point numbers, but also consider complex
numbers, vectors, tensors, or the like), any sort of strings.

All of those possibilities are yours to implement as you like.
I will move on to the back half of compilation in the next part.
That is emitting instead of immediately running.

\section{Exercises}

\subsection{Tests}

Add tests for the new features, specifically running the sample program
from the top of the chapter.

\subsection{Empty Parameter List}

Change the grammar and all supporting code to allow functions that
don't take parameters.
