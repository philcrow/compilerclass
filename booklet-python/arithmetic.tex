\chapter{Arithmetic}

In his book {\it The Definitive ANTLR 4 Reference}, Terence Parr
provides a language for basic arithmetic. I'm going to borrow that
with a few twists.

This version of the language will support addition, subtraction,
multiplication, division, and parentheses for their usual role in
controlling precedence. It will also introduce variables which must
be assigned a value before use, but then can appear anywhere a number
is allowed.

\section{Grammar}

We need to learn quite a few features of ANTLR grammars to expand our
languate in this way. This section details those grammar features.
Later in the chapter we will see how to use the new grammar to perform
arithmetic by implementing a visitor.

\subsection{Arithmetic}

To start, I will present just the arithmetic, leaving variables for
a later section.

I'm going to present the new grammar a bit at a time so we can digest it.
All of this goes in the new Calc.g4.

The first big change is that we need more rules. We want to allow
multiple statements in a program. This requires introducing a rule
for statements and making a program depend on those.

{\footnotesize
\begin{verbatim}
    grammar Calc;

    program: statement+ ;
\end{verbatim}
}

Note that we use the same multiplicity operators for rules as
for regular expressions. So, this reads: ``A program is one or
more statements.''

To start, we will have two types of statements: one for expressions
and one for blank lines. It is not nice to work in a language that
prevents us from having vertical space between sections of interest.
This requires the second new grammar feature: alternatives.

{\footnotesize
\begin{verbatim}
    statement: expression NL
             | NL
             ;
\end{verbatim}
}

The pipe symbol means or (which you might also remember from regular
expressions). This reads as: ``A statement is either an expression
followed by a new line OR a bare new line.''

Note that it is typical to list each alternative on a separate line
and to align the colon and pipes vertically. For extra clarity,
the semi-colon has its own line.

Later we will add a variable assignment statement as an alternative.

Defining what we mean by a statement has introduced two new undefined
terms. This is typical in grammar work. We have to keep defining
until everything has a rule or token definition.

For arithmetic we need the four operations and parentheses. Recall
that precedence matters a lot for this kind of routine Math.

An expression like

{\footnotesize
\begin{verbatim}
    3 + 4 * 5
\end{verbatim}
}

could be computed exclusively left to right. Then, three plus four
is seven, which multiplied by five is 35. But, that isn't what we
would normally do. Instead, multiplication would take precedence.
Thus, as we learned in school, we should multiply four by five to
make twenty, then add three for an answer of 23.

When you design a language, you have full control. It is within your
power to defeat convention. Sometimes that is exactly what you should
do. But, as Larry Wall frequently writes, you should normally abide
by the principle of least surprise. It would be very surprising to
have the rules from school ignored here.

How do we tell ANTLR about our desire for normal precedence?
We list the higher precedence operation first. That is a simple
principle, but it requires some contortion when two or more operations
need to be at the same precedence level, as we are about to see.

{\footnotesize
\begin{verbatim}
    expression: '(' expression ')'
              | expression (STAR|SLASH) expression
              | expression (PLUS|MINUS) expression
              | NUMBER
              ;
\end{verbatim}
}

The first alternative is anything in parentheses.\footnote{This doesn't
actually have to be listed first to win a precedence race. The parenthesis
token is only valid for this rule, so it would win regardless of order.}
Then comes multiplication and division in an alternative together.
They are followed by addition and subtraction in a joint alternative.
At the end is the humble number.

Note that I have chosen to name the operator tokens rather than using
literals. This will make it slightly easier to determine which operation
matched when we have to choose between addition and subtraction or
multiplication and division.

We have our rules. Now it remains to define the tokens.

{\footnotesize
\begin{verbatim}
    NL          : '\r'? '\n' ;
    NUMBER      : [0-9]+ ;
    STAR        : '*' ;
    SLASH       : '/' ;
    PLUS        : '+' ;
    MINUS       : '-' ;
    HS          : [ \t] -> skip ;
\end{verbatim}
}

A newline (NL) is an optional carriage return and a required line feed. This
accounts for the two common line endings. Microsoft Windows uses both
characters. Unix systems, including Macs, use only the line feed.

A NUMBER is still the integer only first thought version from a prior
chapter. If you did the exercise to improve it, use that instead.
Also, consider moving to floats here.

The arithmetic tokens (STAR, SLASH, PLUS, and MINUS) are just names for
the characters.

Finally, HS is horizontal space. This is the answer to an exercise
from a prior chapter. This character class includes spaces and tabs.
The special dash arrow syntax allows us to immediately act within the
lexer. By choosing skip as that action, the spaces we don't care about
a discarded.\footnote{If you need to use spaces, you have to be more
careful. You would need them for a language like python where they matter,
or if you want to emit source code that looks the same as the original,
possibly for error message reporting. Ask the internet what to do in
those cases.}

\subsection{Variables}

To introduce variables, we need only three things: a new statement
alternative for assigning to them, a new expression for using them
in other expressions, and a token controlling their legal name pattern.

{\footnotesize
\begin{verbatim}
    statement: ...
             | ID '=' expression NL
\end{verbatim}
}

An assignment statement is a variable name (ID), a literal equal sign,
any expression, and a newline.

{\footnotesize
\begin{verbatim}
    expression: ...
              | ID
\end{verbatim}
}

Any legal variable name can be used directly in any other expression
as if it were a literal number. Note that that grammar cannot enforce
our rule that a variable needs to have a previous assignment before use.
That is a semantic rule. We will need to enforce that after the parser
builds the abstract syntax tree (AST).

The token for a variable name can be simple.

{\footnotesize
\begin{verbatim}
    ID : [a-z]+ ;
\end{verbatim}
}

This allows any lowercase letters. Most languages are more generous.
Look for an exercise to address this.

\subsection{Labels}

We can use the grammar from the last sections, but it will require
quite a bit of extra work. It would generate a single visitStatement
method and a single visitExpression method. We would need conditional
logic to figure out which rule matched before we could do our computations.

To help us, ANTLR provides labels. They look like comments, but
are requests for extra methods with supplied names. At the end of
each alternative, we add a pound sign (\verb+#+) and the name we want.
I'll show the complete grammar now, notice the labels.

inject Calc.g4 from (stages/arithmetic)

There are two other labels. In the expression
alternatives\footnote{Rule alternatives are sometimes called productions.
That is because in tools like yacc there would be code to make AST nodes
after the match rule. ANTLR is producing those AST nodes for us, so it
is still a production, just an automated one.} for multiplication/division
and addition/subtraction, I labeled the operator match fragment `op`.
That will generate a useful attribute of the context objects with that
name. We'll see that this makes it easier to pick between the two
operations sharing each precedence level.

\section{Visitor Implementation}

Now that we have the grammar, we can generate the lexer, parser, and
visitor base class.

{\footnotesize
\begin{verbatim}
    antlr Calc.g4
\end{verbatim}
}

We need the same three code files as for the single digit adder:
a test, a driver, and the visitor. Since the grammar has the same name,
we can use the driver from the single digit adder without modification.
We can also use the test class, but we will want to expand it to cover
the new language features. Thus, I will only show the new visitor.
You need to write tests to cover every feature of the language.

\subsection{Numbers}

In a visitor, the methods return something useful. In our case that
will be the calculation result for expressions. If we were continuing
compilation with semantic analysis, optimization, and emission, we would
need to return nodes that formed a tree or digraph.

One of the nice features of ANTLR is that you can work in stages.
This fits our minimum viable product/enhancement approach. Start with
the minimum. We need only the visitPrint and visitnNumber to make
the first test pass.

Here is the test class

{\footnotesize
\begin{verbatim}
    class TestVisitor(unittest.TestCase):
        def test_number(self):
            program = "4\n"
            tree = self.get_tree(program)
            visitor = Visitor()
            answer = visitor.visit(tree)
            self.assertEqual(4, answer)

        def get_tree(self, program):
            input_stream = InputStream(program)
            lexer = CalcLexer(input_stream)
            stream = CommonTokenStream(lexer)
            parser = CalcParser(stream)
            tree = parser.program()
            return tree

    if __name__ == '__main__':
        unittest.main()
\end{verbatim}
}

This is similar to the one we saw when testing single digit adding.
But, I have made a helper (\verb+get_tree+) to handle the standard compilation
steps that lead to the AST.

Notice that the program in \verb+test_number+ is just a four and a newline.
We required the newline statement terminator in our grammar. ANTLR
won't recognize the statement without it. When we visit the tree,
we will receive the return value of the visitProgram method. We don't need
to implement that. The inherited one will do nicely.

When the visitor sees our program, it will first capture the number,
then the new line. From the grammar, this will become a tree like
this\footnote{This tree uses the labels for statement and
expression alternatives.}:

{\footnotesize
\begin{verbatim}
    program
       |
     print    
    /     \
   number  NL
     |
   NUMBER
     |
     4
\end{verbatim}
}

Each of the rule alternatives in this tree will have a method in
the generated CalcVisitor. That's why we need to implement visitPrint
and visitNumber.

To make this test pass, we need only visitPrint and visitNumber in
our Visitor.py:

{\footnotesize
\begin{verbatim}
    from CalcVisitor import CalcVisitor
    from CalcParser import CalcParser

    class Visitor(CalcVisitor):
        def visitPrint(self, ctx:CalcParser.PrintContext):
            answer = self.visit(ctx.expr())
            print(answer)
            return answer

        def visitNumber(self, ctx:CalcParser.NumberContext):
            answer = int(ctx.NUMBER().getText())
            return answer
\end{verbatim}
}

This visitNumber shares some features with visitProgram from our single
digit adder. In particular, it invokes NUMBER as a method on the context
object. In this case, there is only one number in the rule alternative,
so there is no argument. Calling getText returns the string that matched.
Python provides the int builtin function to convert that string to an
actual number. If you expand the definition of number to allow
decimals, switch from calling into to calling float.

I collected the answer on one line, then returned it on a separate line.
That allows me to add debugging with the answer as needed.

In visitPrint, I do the actual printing. But, I also return the value
so tests don't have to trap the console output to make their assertions.

\subsection{Operations}

Now that we are handling number expressions, we can move up to
calculating. Let's look at visitAddSubtract.

{\footnotesize
\begin{verbatim}
        def visitAddSubtract(self, ctx:CalcParser.AddSubtractContext):
            left = self.visit(ctx.expression(0))
            right = self.visit(ctx.expression(1))
            if ctx.op.text == '+':
                answer = left + right
            else:
                anwer = left - right
            return answer
\end{verbatim}
}

There are two expressions involved in these
operations\footnote{which is why we call them binary operations}.
If you look in AddSubtractContext in the grammar, you can see that
the expression method takes an argument. As with the single digit
adder, call that with zero and one to get the two operands.
The visit will terminate at either a number or an id expression,
both of their visitors return integers ready for computations.

In the grammar, I labeled the operator `op`. That created an attribute
of the context class with that name. It is an object of type
CommonToken, which has a text attribute. That tells us which symbol
matched. We know we are either adding or subtracting, so we can
test for plus sign. If it is plus, we add, otherwise subtract.

Again, I stored the calculation result in answer, so that I can
add debugging as needed.

I won't show visitMultiplyDivide. You can work that out for yourself.

\subsection{Variables}

There are two visit methods for variables: visitAssign and visitId.
The first one needs to compute the right hand side expression and
store it. The second one needs to look up that answer when it appears
in a subsequent expression. Start with a test

{\footnotesize
\begin{verbatim}
        def test_var(self):
            program = "x=3\nx+4\n"
            tree = self.get_tree(program)
            visitor = Visitor()
            answer = visitor.visit(tree)
            self.assertEqual(7, answer)
\end{verbatim}
}

This also tests add, so you could remove the separate add test if you like.
Note that I need to follow the language syntax carefully. There must
be newlines between the statements. The variable must be declared first.

In order to store the values of assigned variables, we need a symbol
table. At first this will be very simple. It will just be a python
dictionary\footnote{called a Map in java or a hash in perl}. Later
we will extend our notion of symbol tables to handle block and function
scoping.

Every method in the visitor should be able to see the symbol table.
We need to make it an attribute of the Visitor class. We do this by adding
a constructor.

{\footnotesize
\begin{verbatim}
    class Visitor(CalcVisitor):
        def __init__(self):
            self.symbol_table = {}
\end{verbatim}
}

In python constructors are called \verb+__init__+. They can receive
addtional parameters, but the first one is the object under construction.
To add attributes to class instances, we assign them to self.NAME.
The value here is an empty dictionary.

In visitAssign, we add or update values in \verb+symbol_table+:

{\footnotesize
\begin{verbatim}
        def visitAssign(self, ctx:CalcParser.AssignContext):
            name = ctx.ID().getText()
            value = self.visit(ctx.expression())
            self.symbol_table[name] = value
            return value
\end{verbatim}
}

First, lookup the ID from the context. That's the variable's name.
Then, visit the expression in the context to calculate that value.
Store that in the dictionary. No one is using the return value of
visitAssign, but I return the value anyway out of
habit.\footnote{In strongly typed languages, like Java, all the
visit methods must return the same type.}

Finally, using variables is just looking up their values in the
\verb+symbol_table+.

{\footnotesize
\begin{verbatim}
        def visitId(self, ctx:CalcParser.IdContext):
            name = ctx.ID().getText()
            answer = self.symbol_table[name]
            return answer
\end{verbatim}
}

Again, the ID token in the context has the name. We just need to
dereference that in the dictionary to know the value to return.
If the variable is not there, bad things will happen (python
will raise a KeyException). We should help the user with a good
error message.

\section{Exercises}

\subsection{Finish It}

Fill in the rest of the methods not shown above so that you can
run a sample program like

{\footnotesize
\begin{verbatim}
    x = 4
    3 + 4 * (x - 2)
\end{verbatim}
}

\subsection{Unary Negation}

Expand the grammar to include a unary negation expression. Implement
its visit method so these can work

{\footnotesize
\begin{verbatim}
    -3*6
    14--5
\end{verbatim}
}

\subsection{Semantic Complaint}

Currently, when visitId can't find the variable, python raises
a KeyError. Improve the error reporting. Take it as far as you dare.
It would be best to terminate the run while saying something like

{\footnotesize
\begin{verbatim}
    Use of unassigned variable 'name' at line 2.
\end{verbatim}
}

\subsection{Tests}

Add tests to cover subtraction, multiplication, division, variable
assignment and use, and parentheses. Write little test programs
which test one feature each. Keeping tests small and focused makes
them easier to maintain. The last thing you want is to have colleagues
turn off or delete tests because they are annoying to maintain.

\subsection{Variable Names}

Improve the ID token so it is at least as generous as Java or Python.
That means allowing uppercase letters, but also numbers after the first
position. It also allows underscores. In Javascript, you can use dollar
sign anywhere in the name. Make your choice and define that token.
