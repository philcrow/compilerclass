\chapter{Language Exercises}

Even though there are mounds of features our language from the prior
chapters does not have, it actually pretty sophisticated. Many tasks
in computing are done with languages that don't have functions,
loops, or even variables.

The odds that you will use the concepts and tools from the early
part of this book to work on a production compiler are not great.
But, there are lots of problems which come up in the day to day
life of the working developer that benefit from these ways of
thinking.

This chapter presents some exercises you can do to practice grammar
based parsing and AST use.

\section{ASCII Art}

When I took typing in junior high, our teacher gave us fun handouts
(well, she thought they were fun). Each line on the handout was one
line to type. Everything on the line was a number followed by a letter
to type. When you followed instructions, you got ASCII art.

Example Input

3 45-
1
5 5-4 5-4 5-4 5-4 5-
1
3 45-

(Note that the lines that start with ones have a single trailing space.)

Example Output for that input

   ---------------------------------------------

     -----    -----    -----    -----    -----

   ---------------------------------------------

[ we need a way to provide examples here ]

Make a grammar that parses this input and a corresponding visitor
that can emit the output.

\section{Story Board}

I like to use dot from the graphviz project. Part of the reason is
beacause it is text based. That means you can easily generate input
for it.

Consider a little language to diagram a play or screen play.

Example Input

    act {
        scene meetcute "BOY meets GIRL, but he is rude"
        scene ...
    }

Example Output

...

Convert the input into a dot directed graph, which is a story board..
Either install graphviz or use https://sketchviz.com/new to see
your results.

\section{Code Generator}

There are many tools that generate code. Every IDE has a set of
generators. These can save a lot of typing, especially in languages
with a lot of boilerplate. Get in on the generating act yourself.

Example Input

    class Parent {
        int x
        String name
    }

    class Child from Parent {
        String nickname
    }

From this generate classes with proper inheritance. Provide:
    attributes
    a constructor to accept and store those attributes
    accessors for them

Use any language that supports class based inheritance are you target.

\section{Emitting Target Code}

Choose a version of the language we developed in prior chapters.
Emit python for programs in the language. This involves writing a new
visitor which spits out python instead of running the program.
You'll know your solution is working when the python program runs
with the same results as the original (although it will probably chat less).

Use a version of the lanugage which has at least fully functioning
arithmetic. Start with that. Then add variables, flow of control (if
and while), and -- if you dare -- functions.
