\chapter{Flow of Control}

A language that always runs top to bottom through all the same lines
is not that useful. We need logic. Specifically, we need if with
else to choose a code path and while for loops to repeat steps.

\section{Logic with If and Else}

\subsection{Goal}

Our if will allow a program like this sample:

{\footnotesize
\begin{verbatim}
    hours = 42
    rate = 10

    if (hours > 40) {
        pay = 40 * rate + (hours - 40) * 3 / 2
    }
    else {
        pay = hours * rate
    }
\end{verbatim}
}

\subsection{Grammar Additions}

Even though this spans several lines, we will treat the whole if structure
as a single statement.

{\footnotesize
\begin{verbatim}
    statement: ...
             | 'if' '(' condition ')' block else?  # if
\end{verbatim}
}

The if and parentheses tokens are literal. We need to define
condition, block, and else. These need new rules.

{\footnotesize
\begin{verbatim}
    block: '{' statement* '}' NL? ;
\end{verbatim}
}

This leverages the statement definiion we already have. That makes
the block contents a mini-program. The newline is optional so the
user can cuddle braces if they like, as in

{\footnotesize
\begin{verbatim}
    if (condtion) {
        ...
    } else {
    }
\end{verbatim}
}

Note that we are explicitly allowing empty blocks. Not all languages
do that. In Python you need a statement in every block. It provides
the builtin `pass' in case you need to do nothing. That's a descendent
of the old assembly NOOP.

Once we have a defined block, else can use that.

{\footnotesize
\begin{verbatim}
    else: 'else' block ;
\end{verbatim}
}

Remember that we already made else optional for the if statement.
Here we are saying what it will have to look like, if it is present.

Lastly, we need a rule for condition.

{\footnotesize
\begin{verbatim}
    condition: expression CONDITIONAL expression ;
\end{verbatim}
}

This will allow us to compare any two expressions. This becomes vastly
more complicated if the language has multiple types. Then, you need
to impute the proper mode of comparison based on the types involved.
Some languages, notably Perl, offers separate conditional operators
for numbers and strings. That lets the programmer dictate the type coercion.

The conditional operators are defined as a token.

{\footnotesize
\begin{verbatim}
    CONDITIONAL : '==' | '!=' | '<=' | '<' | '>=' | '>' ;
\end{verbatim}
}

\subsection{Implementation}

As test programs get longer, it is nice to move to python's here documents
to specify them. Here is my if test (in TestVisitor.py):

{\footnotesize
\begin{verbatim}
        def test_if(self):
            program = '''
            hours = 42
            rate = 10

            if (hours > 40) {
                pay = 40 * rate + (hours - 40) * 3 / 2
            }
            else {
                pay = hours * rate
            }
            '''
            tree = self.get_tree(program)
            visitor = Visitor()
            visitor.visit(tree)
            self.assertEqual(430, visitor.symbol_table["pay"])
\end{verbatim}
}

Notice that I also peek in the symbol table at the variable pay to
check that the calculation went as expected.

We need to visit methods to make this work: visitCondition and visitIf.
We can't easily test visitCondition by itself. So, we will use tests
with if statements to reach it.

{\footnotesize
\begin{verbatim}
        def visitCondition(self, ctx:CalcParser.ConditionContext):
            ''' returns 0 for false, 1 for true'''
            left = self.visit(ctx.expression(0))
            right = self.visit(ctx.expression(1))
            operator = ctx.CONDITIONAL().getText()

            if operator == '==':
                return 1 if left == right else 0
            if operator == '!=':
                return 1 if left != right else 0
            if operator == '<=':
                return 1 if left <= right else 0
            if operator == '<':
                return 1 if left < right else 0
            if operator == '>=':
                return 1 if left >= right else 0
            if operator == '>':
                return 1 if left > right else 0
\end{verbatim}
}

Python doesn't have a case construct, so I need a series of if statements.
I could use else, but I'm returning from each, so that would be
redundant. Retrieving the values to compare is the same as for
visitAddSubtract. The context object provides the operator via the
CONDITIONAL method.

Notice the pydoc comment. I'm imposing a constraint: all my visit
methods will return integers. It is typical that all visit methods
return the same type. In languages with stronger explicit typing, like java,
this is an actual requirement.

Once we can compute the truth of the condition, implementing if is not
complicated.

{\footnotesize
\begin{verbatim}
        def visitIf(self, ctx:CalcParser.IfContext):
            truth = self.visit(ctx.condition())
            if truth == 1:
                self.visit(ctx.block())
            else:
                self.visit(ctx.else_())
\end{verbatim}
}

If visiting the condition returns one, visit the block. Otherwise,
visit the else block. Note that ANTLR has appended an underscore to
the name, since else is a python keyword.

We don't even need to implement visitElse. The default provided in
CalcVisitor will visit its children. And, we don't need to worry about
the case where these is no else block. The inherited visitor handles
that too.

\section{While}

\subsection{Goal}

We want to be able to loop with the usual while keyword.

{\footnotesize
\begin{verbatim}
    n = 10
    while (n > 0) {
        n = n - 1
    }
\end{verbatim}
}

It's a pity we can't actually print ``Liftoff!'' at the end.

Since I have shown how to implement if in detail, I'll leave while
completely to you. The basics are the same, the implementation just
needs to check the conditional in its own loop.

\section{Exercises}

\subsection{While}

Implement the while loop.

\subsection{Tests}

Make tests for all variations. In particular, make a test which reaches
an else block.

\subsection{Else If}

Add an unlimited number of optional else if clauses to the grammar.
Fix the visitIf method to handle them correctly.

\subsection{Keyboard Input}

It isn't fun to have to hard code values in the programs. Implement
a new statment

{\footnotesize
\begin{verbatim}
    take hours = 'Enter hours worked->'
\end{verbatim}
}

In its visitor, issue the prompt to the console, accept the users
input, and store that in the named variable.

\subsection{Output}

Add a print statement that allows arbitrary text to go to the keyboard.
Bonus feature: allow interpolation like Python's formatted strings
with interpolated variables in braces or something similar.
