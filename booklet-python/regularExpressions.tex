\chapter{Regular Expressions}

One of the key failings of our digit adder from the last chapter is that
it only handles one digit numbers. A first improvement was an exercise:
expand to handle more integers. To do this we need to learn a bit about
regular expressions.

\section{Multiplicity}

We often want to allow a matched item to be repeated. Here we want more
digits. In other case we might want multiple statement in a block,
or multiple parameters in a function definition. Regular expressions
provide this. There are three basic operators to help us. We can also
use these in grammar rules.

When we want one or more of something, we use a trailing plus (+). So,

    [0-9]+

is a naive way to solve the single digit problem. Now we can have any
integer.

We don't need the other multiplicities now, but it makes sense to talk
about them together.

Sometimes we want something to be optional, but repeatable. One example
is else if after an if construct. When you use conditional logic, you don't need
any else or else if blocks. But, you can have as many of the latter
as you like. For this we use an asterisk (*). Suppose that we have
already defined what we mean by elseif, using a grammar rule with that name.
Then we can allow zero or more of those with

    elseif*

Finally, sometimes we want something to be optional. That is, we want
zero or one. For that we use a question mark (?). For instance, we could
allow negative numbers with an optional leading minus sign.\footnote{
This is not the ideal way to account for negative numbers. We should use
a unary negation operator instead.}

    '-'? [0-9]+

To summarize

\begin{tabular}{l l}
    Operator &      Meaning \\
    +        &      One or more \\
    \verb+*+ &      Zero or more \\
    ?        &      Optional (zero or one) \\
\end{tabular}

\section{Alternatives}

We can use

    NUMBER: [0-9]+;

to allow multi-digit integers. But, that allows things we probably don't
want, like 00001. In some languages a leading zero forces the number to
be octal. We don't want people to think that is happening in our language.

If we want a more traditional number, we need more than one expression.
We need to be able to pick between expressions. In regular expressions,
we use a pipe symbol (\verb+|+) to indicate choose one of these. This allows us
to have zero and non-zero numbers, but not leading zeros.

{\footnotesize
\begin{verbatim}
    NUMBER: '0' | [1-9] [0-9]* ;
\end{verbatim}
}

This says: A NUMBER is a literal zero OR any digit 1-9 optionally
followed by any digit. Note that the notion of `and' is achieved by
setting two expressions side-by-side, but the order is enforced.

\section{Excercises}

\subsection{Rational Numbers}

In the last chapter I used the python int function to convert our numbers
from strings for calculation. I could have used float instead. Then,
we could use rational numbers instead of integers. This seems like
a good idea. Write a definition of NUMBER which allows for positive
rational numbers. Hint: alternatives are important here and the first
one is zero by itself. Hint 2: consult www.json.org for
a railroad track diagram of number.

\subsection{Variable Names}

Jump ahead in your thinking to imagine our language has variable.
Write a token rule to define them. What symbols will be allowed in
variable names? Can all of those be in the first position?
Note that character classes can any number of ranges, which can
include letters like a-f. They can also include single characters.
