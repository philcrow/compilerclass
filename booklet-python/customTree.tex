\chapter{Internal Representation}

To this point, we have used the awesome AST built for us by ANTLR.
While it is amazing how far we have gotten with it, there are limits.
We need to build our own internal representation, if we want to
emit assembly. In other words, to complete the back half of the
compiler we need a new IR.

We will again start with a minimal approach to introduce the concepts.
The internal representation will be a graph of nodes. Ours will
be a tree.\footnote{Some optimization schemes compress this into
a directed graph when two subtrees are identical.}

Each node in this tree will need to be an object descended from the same type.
Traditionally, they would all share the same attributes. But, since
we have objects instead of more primative structures, we don't need to
overload like we would in C.

Each node type will need a constructor for the visitor to call.
They will also need a common method that does something useful.
Our goal is one which emits a target language. But, it could also
draw something useful for debugging, etc. It could even run the
program as we have been doing. In fact, I will start with two
methods in each node: getIntValue (for primitives like
numbers) and act (to interpret like we have been doing). Later,
I will add an emit method to generate assembly output.

Once we have a tree of these nodes, we could implement other tools.
We could write a set of analyzers which walk the tree to enforce
language rules, like missing declarations. When we interpret, those
are run time errors. With an analyzer they can be part of
compilation. This reports the error earlier, before part
of the code runs. Analyzers are beyond our scope. We will generate
errors like this when we notice them as we emit assembly.

Optimizers walk the tree to build an optimized replacement. Since
they are making the same structure, just with different specific
nodes, optimizers can be mixed and matched with each other and
with analyzers. Again, optimizing is an advanced topic beyond our scope.

To distinguish this new tree from the AST built during parsing,
we call this a decorated tree or an internal representation (IR).

\section{Nodes}

Our Node class is quite simple.

{\footnotesize
\begin{verbatim}
    class Node:

        def act(self):
            pass

        def getIntValue(self):
            pass
\end{verbatim}
}

These are the methods each node could implement: act and getIntValue.
The methods implemented with pass statements in the Node parent class
will work for the children that don't need one of the methods.

To start consider the simplest program in our language:

{\footnotesize
\begin{verbatim}
    4
\end{verbatim}
}

When we interpret, this prints `4'.
When this parses, the AST will look like this:

{\footnotesize
\begin{verbatim}
    program
       |
     print
       |
     number
       |
       4
\end{verbatim}
}

This needs the first three node types: ProgramNode, PrintNode, and NumberNode.
I'll show them in reverse order, because they rise in complexity in that
order.

\subsection{NumberNode}

{\footnotesize
\begin{verbatim}
    from Node import Node
    class NumberNode(Node):
        def __init__(self, value):
            self.value = value

        def getIntValue(self):
            return self.value
\end{verbatim}
}

This is a leaf node. It only needs to implement getIntValue.
Its parent will take care of act (and emit when we add it).
All this node needs to do it keep track of the value of its
number and return it upon request.

\subsection{PrintNode}

{\footnotesize
\begin{verbatim}
    from Node import Node
  
    class PrintNode(Node):
        def __init__(self, valueNode):
            self.valueNode = valueNode

        def act(self):
            print(self.getIntValue())
            return self.getIntValue()

        def getIntValue(self):
            return self.valueNode.getIntValue()
\end{verbatim}
}

The statement production for a bare expression will become a PrintNode.
This needs to store the child node during construction. If asked
for the int value, it delegate to that child. Similiarly, to act
during interpretation it looks up the value, prints it to stanard output,
and returns it.

\subsection{ProgramNode}

Finally, the most complicated of our nodes so far will keep a list
of statements and run them in order when asked to act.

{\footnotesize
\begin{verbatim}
    from Node import Node
  
    class ProgramNode(Node):
        def __init__(self, statements):
            self.statements = statements

        def act(self):
            for statement in self.statements:
                statement.act()

        def getIntValue(self):
            return 0;
\end{verbatim}
}

Note that the whole program does not have a useful value to deliver.
We hope no one calls getIntValue on this. If they do, they just get
a zero.

To act on the statements, we just loop through them once in order
asking each of them to act.

\section{Making an IR}

Previously our visitor interpreted the program. Now we will instead
build a tree of Node objects. If the caller wants to interpret the
program, they will call act. Later, they could instead could call emit
to produce assembly.

{\footnotesize
\begin{verbatim}
    from CalcVisitor import CalcVisitor
    from CalcParser import CalcParser
    from ProgramNode import ProgramNode
    from PrintNode import PrintNode
    from NumberNode import NumberNode

    class NodeVisitor(CalcVisitor):
        def visitProgram(self, ctx:CalcParser.ProgramContext):
            statements = []
            rawStatements = ctx.statement()
            for rawStatement in rawStatements:
                statementNode = self.visit(rawStatement)
                if statementNode is not None:
                    statements.append(statementNode)
            return ProgramNode(statements)

        def visitPrint(self, ctx:CalcParser.PrintContext):
            value = self.visit(ctx.expression())
            return PrintNode(value)

        def visitNumber(self, ctx:CalcParser.NumberContext):
            value = int(ctx.NUMBER().getText())
            return NumberNode(value)
\end{verbatim}
}

It is easier to begin reading at the bottom. The visitNumber method
begins in the same way as the interpreting version, by looking up
the text the parser found.\footnote{Remember that there is no reason
to guard against conversion errors here, since the parser would
only bring us here if it found a valid number.}

Then, instead of returning that directly, we make a new
NumberNode with that as its value.

The parent of this number in our simple program is a print statement.
The visitPrint method finds its value by visiting the expresssion, as
it did while interpreting. But, now two things are different. First,
the result of calling visit on the child expression is a Node of some
kind. Second, we need to make a PrintNode to house that child Node.

The tree walk is really directed in visitProgram. That asks for
the raw statements found by the parser and loops over them.
For each one, it calls visit receiving whatever Node type that
produces. If the statement makes no node (think blank lines)
it does not store the result.\footnote{If you are going to emit
the original source code, you have to capture those too.}
If the Node is not None, it goes into the list of statements
which are given to the ProgramNode constructor.

Note again that calling this visitor only returns the ProgramNode
which contains its list of child statements. The caller needs
to call act to gain the original behavior or emit to produce assembly.

\section{Testing the IR}

In software we often hear the terms white box and black testing.
Where that latter certainly implies that we cannot see inside whatever
we are testing. My problem with these as opposites is that white
boxes are also opaque. I prefer to use the terms transparnet and opaque
testing. In this case, we will use transparent testing. This means
that the test will use intimate knowledge of the code being tested.

Here is the new test module to transparently test the NodeVisitor
presented in the prior section.

{\footnotesize
\begin{verbatim}
    from antlr4 import *
    from CalcLexer import CalcLexer
    from CalcParser import CalcParser
    from NodeVisitor import NodeVisitor
    import unittest

    class TestNodeVisitor(unittest.TestCase):
        def test_number(self):
            program = "4\n"
            tree = self.get_tree(program)
            visitor = NodeVisitor()
            answer = visitor.visit(tree)
            self.assertEqual(1, len(answer.statements))
            print_statement = answer.statements[0]
            self.assertEqual(4, print_statement.getIntValue())

    def get_tree(self, program):
        input_stream = InputStream(program)
        lexer = CalcLexer(input_stream)
        stream = CommonTokenStream(lexer)
        parser = CalcParser(stream)
        tree = parser.program()
        return tree

if __name__ == '__main__':
    unittest.main()
\end{verbatim}
}

The \verb+get_tree+ method is the same as in our prior tests. It takes
us through the parsing step which builds the AST.

The \verb+test_number+ method call \verb+get_tree+ to obtain the AST
as before. Then it visits the tree. In earlier tests we could then
examine the result directly for numeric answers. Here we receive
a ProgramNode. First, I decided to check that it contains the correct
number of statements, one in this case. Then I peeked inside
the ProgramNode to retrieve the PrintNode and tested its value
which should be four. This is transparent testing, because I have
used node's internal methods at will. This is a typical approach
for a unit test.

\section{Completing the IR}

We now have a working and tested IR for a subset of our language.
To return to the interpreter we had before, we need to implement
a lot more visit methods. Most of those will need to construct
Node objects.\footnote{Some don't need to do that, like parentheses.
Again, if you needed to regenerate the original program you would
need a node for those too. But for assembly emitting or interpreting
you don't.}
