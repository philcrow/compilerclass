\chapter{Internal Representation}

To this point, we have used the awesome AST built for us by ANTLR.
While it is amazing how far we have gotten with it, there are limits.
We need to build our own internal structure in order analyze, optimize,
and emit full programs. In other words, to complete the back half of the
compiler.

We will again start with a minimal approach to introduce the concepts.
The internal representation will be a graph of nodes. Ours will
be a tree. Some optimization schemes compress this into
a directed graph when two subtrees are identical.

Each node in this tree will need to be an object of the same type.
There will be subtypes, but they must share an ancestor.
Traditionally, they would all share the same attributes. But, since
we have objects instead of the more primative structures we would
be stuck with in a language like C, we don't need to overload in that way.

Each node type will need a constructor which the visitor will call to
create one. They will also need a method that does something useful.
Normally, that emits target lanugage output. But, it could also
draw something useful for debugging, etc. It could even run the
program as we have been doing.

Analyzers walk the tree to enforce language rules that the parser
can't, like missing declarations. Optimizers walk the tree to
build an optimized replacement. Since they are making the same
structure, just with different specific nodes, optimizers can
be mixed and matched.

To distinguish this new tree from the AST built during parsing,
we call this a decorated tree or an internal representation (IR).

\section{Nodes}

