\chapter{Getting Started}

There are two parts to beginning work on compilers. One is understanding
the goal. The other is installing the tools. This chapter starts with
an overview of compilation goals and ends with tool installation advice.

\section{What are We Trying to Do?}

Compiling is translating a human readable source code language
into something else called the target language.

\subsection{gcc}

Traditionally, the target language is assembly. The most famous
current compiler in this tradition is gcc. Given a C program,
it will generate assembly, then quietly assemble and link that, so you
receive an executable. How portable will the executable be? Not very.

{\footnotesize
\begin{verbatim}
    #include <stdio.h>

    int main(int argc, char *argv[]) {
        printf("Hello %s\n", "Rob");
        return 12;
    }
\end{verbatim}
}

Compile and run:

{\footnotesize
\begin{verbatim}
    gcc hello.c -o hello
    ./hello
    echo $? # to see the exit status
\end{verbatim}
}

Learn more by visiting: https://gcc.gnu.org/.

From the website you can see that gcc is actually more than a c compiler.
That's just its most famous feature.

\subsection{dot}

There are lots of other related tools which don't generate assembly,
many of which don't make anything that actually runs. One of my favorites
is Dot (see https://graphviz.org/). It takes a descriptive source
language and generates image output in formats like png, jpeg, gif,
pdf, and others.

If you install graphviz on your system, you can save this as family.dot
in your favorite text editor:

{\footnotesize
\begin{verbatim}
    digraph { Mom -> Me }
\end{verbatim}
}

To compile it, run:

{\footnotesize
\begin{verbatim}
    dot -Tpng -o family.png family.dot
\end{verbatim}
}

Then you will have png output in family.png showing this tiny excerpt
of my family tree.

You can read all about the graphviz project and its tools at:
https://graphviz.org/.

If you don't want to install graphviz on your system, you can
still play with it on a site like: https://sketchviz.com/new.

\subsection{TeX}

TeX is a language for typesetting written by Donal Knuth (who also
wrote a famous set of books on the art of programming).
You can write macros in it. Leslie Lamport
wrote a whole bunch of really useful ones which he called LaTeX. In TeX,
you type commands that describe documents and the contents of those
documents into a text editor. Here is an example:

{\footnotesize
\begin{verbatim}
    \documentclass{article}
    \title{A Sample}
    \author{Phil Crow}
    \begin{document}
    \maketitle

    This is a simple sample of a LaTeX article.

    \end{document}
\end{verbatim}
}

Save that as article.tex and run

{\footnotesize
\begin{verbatim}
    pdflatex article.tex
\end{verbatim}
}

to make a PDF. This is my preferred mode of PDF generation.

You can learn to use TeX, and how to install it on your system by visiting:
https://www.latex-project.org/.

\subsection{Lilypond}

Two bright boys, Han-Wen Nienhuys and Jan Nieuwenhuizen, who met in a youth
orchestra thought it would be
cool to use TeX to engrave sheet music. They developed lilypond,
which long ago gave up on using TeX. The syntax still looks a lot
like TeX.

Given a text file called scale.ly with this content:

{\footnotesize
\begin{verbatim}
    \score{
        {
            c' d' e' f' g' a' b' c''
        }
        \layout{}
        \midi{}
    }
    \version "2.18.2"
\end{verbatim}
}

To generate sheet music as a PDF type:

{\footnotesize
\begin{verbatim}
    lilypond scale.ly
\end{verbatim}
}

Since the layout and midi commands are there, you also get scale.midi
with a grand piano playing the music.

If you don't want to install Lilypond on your system, you can play
online at various sites like: https://www.hacklily.org/.

As you may have noticed, I am rather obsessed with tools that take
source from a text file and produce something useful.

\subsection{Typescript}

JavaScript is one of my favorite languages. Since the browser takes
care of running it, I can get all kinds of spectacular behaviors
for users with it, often in only a few lines. It also reminds me
of other languages I have enjoyed in the past, like Perl.

There are those who have a dislike for languages like this in general
(loosely typed, being the main complaint). Many different groups of
JavaScript haters have developed alternatives.

One of those is Typescript. You write in Typescript, then convert
it to JavaScript, since that is the only language the browsers understand.

Languages that convert one source language to another are called transpilers.

If you want to play with it see

    https://www.typescriptlang.org/play/

You can also install it on your system.

\subsection{Sass}

CSS is the language which controls the appearance of web pages. It is
also often maligned. CSS is descriptive, that is it has no conditionals
or loops. As with JavaScript, many have tried to improve it by making
a super-language which is more expressive, but can be transpiled to CSS.
The one I have heard the most about is Syntactially Awesome Style Sheets
(Sass).

To read more about it, see

    https://sass-lang.com/guide

\subsection{Virtual Machines}

Returning to languages that run, we find that an executable, which only
runs on the same system as the compiler which built it, is not the only
option. Some languages target a virtual machine.

In the late 1970s computers were a hot new thing. Size was coming down.
You no longer needed a room (with its own air conditioner) to house one.
The problem was that everyone was building them, each using a different
architecture. Infocom wanted to produce games. The problem was how to
make them run on whatever system the customer bought.

Joel Berez and Marc Blank answered that question by developing a
device indpendent virtual machine called the Z-Machine. For each new
computer that came out, a small number of developers at Infocomm
would port that to the new platform. All the games compiled to its
format. Once the port was finished the whole line of games would
work on the new system.

There were virtual machines before this. And, there is a famous one
that came later called the Java Virtual Machine. The idea is always
the same. Compile the source code to the format of the virtual machine.
Port the virtual machine to all platforms of interest. All the programs
run on all the platforms.

\subsection{Summary}

Compilers take source language from a text file that humans can read
and write. They turn that into a target language or format, usually
one that humans would find hard to read, but that the computer can
render or run.

\section{What Tools will We Use?}

Our goal in this course is to implement a small language, or really
a series of small lanuages of increasing utility (and corresponding
complexity). To do this, we will describe our languages with ANTLR4.

To follow along, you need to install a Java Development Kit (JDK),
python, ANTLR4, and its python runtime. You don't have to use these tools.
I have learned a lot about compilers by reading books and translating
their examples from the language in the book to the language I prefer.
You could do the same, but then I won't spend time worrying about you.

Visit https://openjdk.org/ if you need to install Java.

Visit https://www.antlr.org/ to install ANTLR4. On its download page
are careful instructions for each language. Follow the ones for yours.

Most of the work in this course will feature an interpreter rather than
a full compiler. An interpreter doesn't emit an executable or byte codes,
but runs immediately after parsing the input. If you like, you can
replace the interpreter with a new back end to finish a full compiler.
I discuss this in the book, but it is an advanced path for either
more advanced students or a course longer than one semester.

(Actually, a geniune interpreter reads one statement at a time from
the source and executes it. Bash is like that. My definition is
a bit broader, but the feel is the same. It does not emit an artifact
that you later run, it runs right away. But, it does compile, so it
can have forward references to functions, for instance. And it will
notice syntax errors before they cause a fatal run-time error.)

If you want to generate assembly, you need an assembler. Good news:
gcc has that inside it. That's one choice. I also have a little
pseudo-assembler I wrote to give a feel for assembly without worrying
about actual chip instructions. I'll show how to target these with
an emitter.

There are lots of topics that won't fit in this book. These include
how to implement a scanner or parser, finite automata theory, directed
graphs, and optimization. My goal is to build a foundation of understanding
so that students can confidently implement their own parsing programs
and small languages using those. With that understanding, they may choose
to dig into the full details of modern compilers.

Side hint for Java folks: Visit https://gradle.org/ to install gradle.
Find the antlr plugin for that. It works great.
